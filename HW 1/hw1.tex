\documentclass[10pt,letterpaper]{article}
\usepackage[utf8]{inputenc}
\usepackage{lmodern} % removes bitmap font
\usepackage[bookmarks,colorlinks,breaklinks]{hyperref}
\hypersetup{urlcolor=blue, colorlinks=true, citecolor=green!50!black, linkcolor=blue}
\usepackage{multirow}
\usepackage[T1]{fontenc} % For polish hook https://tex.stackexchange.com/questions/75396/how-to-use-polishhook-symbol
\usepackage{amsmath}
\usepackage{amsfonts}
\usepackage{amssymb}
\usepackage{amsthm}
\usepackage{thmtools}
\usepackage{pgffor}
\usepackage{xcolor}
\usepackage[lined,boxed,ruled,norelsize,algo2e,linesnumbered,noend]{algorithm2e}


\usepackage{cleveref}
\usepackage{graphicx}
\usepackage{geometry}
\geometry{verbose,tmargin=1in,bmargin=1in,lmargin=1in,rmargin=1in}
\usepackage{enumitem}
\usepackage{tcolorbox}

\declaretheorem[refname={Theorem,Theorems},Refname={Theorem,Theorems}]{theorem}
\declaretheorem[numberlike=theorem]{lemma}
\declaretheorem[numberlike=theorem]{invariant}
\declaretheorem[numberlike=theorem]{corollary}
\declaretheorem[numberlike=theorem]{definition}
\declaretheorem[numberlike=theorem]{claim}
\declaretheorem[numberlike=theorem]{fact}
\declaretheorem[numbered=no]{remark}

% number sets
\newcommand{\R}{\mathbb{R}}
\newcommand{\N}{\mathbb{N}}
\newcommand{\F}{\mathbb{F}}
\newcommand{\Z}{\mathbb{Z}}
\renewcommand{\S}{\mathbb{S}}%

% probability
\renewcommand{\P}{\mathbb{P}}
\newcommand{\E}{\mathbb{E}}

% redfine some classical notation
\renewcommand{\tilde}{\widetilde}
\renewcommand{\hat}{\widehat}
\renewcommand{\bar}{\overline}

\renewcommand{\O}{\mathcal{O}}

% operators
\newcommand{\median}{\operatorname{median}}

\DeclareMathOperator*{\argmin}{argmin}
\DeclareMathOperator*{\argmax}{argmax}
\newcommand{\poly}{\operatorname{poly}}
\newcommand{\polylog}{\operatorname{polylog}}
\newcommand{\mdiag}{\mathbf{Diag}} % diagonal matrix with vector on diag
\newcommand{\sign}{\operatorname{sign}}%
\newcommand{\nnz}{\operatorname{nnz}}%
\newcommand{\tail}{\operatorname{tail}}%
\newcommand{\dist}{\operatorname{dist}}%
\newcommand{\IGNORE}[1]{}

% matrices
% \mX is \mathbf{X}
\foreach \x in {A,...,Z}{%
	\expandafter\xdef\csname m\x\endcsname{\noexpand\mathbf{\x}}
}
% \omX is \overline{\mathbf{X}}
\foreach \x in {A,...,Z}{%
	\expandafter\xdef\csname om\x\endcsname{\noexpand\overline{\noexpand\mathbf{\x}}}
}

\foreach \x in {A,...,Z}{%
	\expandafter\xdef\csname c\x\endcsname{\noexpand\mathcal{\x}}
}

\newcommand{\ob}{\overline{b}}
\newcommand{\ox}{\overline{x}}
\newcommand{\os}{\overline{s}}

% Reduce paragraph spacing https://tex.stackexchange.com/questions/4891/how-do-i-control-the-spacing-above-a-new-paragraph
\makeatletter
\renewcommand{\paragraph}{%
	\@startsection{paragraph}{4}%
	{\z@}{1.25ex \@plus 1ex \@minus .2ex}{-1em}%
	{\normalfont\normalsize\bfseries}%
}
\makeatother

\def\norm#1{\left\| #1 \right\|}



%\newcommand{\solution}[1]{{\color{blue} #1}}
\newcommand{\solution}[1]{}

\newcommand{\snote}[1]{\textcolor{red}{Sahil: #1}}


\title{Divide \& Conquer\\Problem Set 1 -- CS 6515/4540 (Fall 2025)}
\date{}
\begin{document}
\maketitle
\vspace{-50pt}
\noindent
This problem set is due on \textbf{Thursday, August 28th}. Submission is via Gradescope. Your solution must be a typed pdf (e.g.~via LaTeX) -- no handwritten solutions.


\section{O-Notation}

\subsection{LLM Question}
Use the following prompt in an LLM (like chatGPT/CoPilot/Gemini) and upload the transcript of your session.

``Act like an undergrad with a CS major who wants to understand Big O notation. Now interactively ask me 5 questions (i.e., one by one while waiting for my answers) to improve your (and my) understanding of Big O. In the end, give me all the correct answers and constructive feedback.''

%\paragraph{Question:}
\subsection{Question:}
We are interested in finding \emph{all} constants for which the following $O(\cdot)$ bounds hold.
\begin{enumerate}
\item For which constants $c,d>0$ do we have $n^c = O(d^n)$? Prove your answer.

\item For which constants $c,d>0$ do we have $(\log(n))^c = O(n^d)$? Prove your answer.
\end{enumerate}

Remark: $c,d$ do not need to be integers. In particular, they could be smaller than $1$.

%We are interested in \emph{all} values for which the respective $O(\cdot)$ bounds hold.








\section{Divide \& Conquer Algorithm}

Given a length $n$ array $A$ (you may assume there are only non-negative entries), construct an algorithm that returns the maximum possible product of \emph{contiguous} entries. 
For example when $A=[0.5,2,0.25,2,0.75,2,0.5]$, the maximum product is $3$ from the sub-array $[2,0.75,2]$.

\paragraph{Problem:}
Construct a divide \& conquer algorithm \textsc{MaxProduct}$(A[1...n])$ that solves the above problem. 
%
Your answer must provide the following:
\begin{itemize}
    \item[(a)] A pseudo-code description of your algorithm.
    \item[(b)] A correctness argument. You may provide a proof by induction, but answering the following questions suffices:
    \begin{itemize}
    \item[(a,i)] Why is the base case correct (when $A$ has length 1)?
    \item[(a,ii)] Why does the algorithm return a correct answer, if it has the solution for the left and right half of the array, i.e., $\textsc{MaxProduct}(A[1...n/2])$ and $\textsc{MaxProduct}(A[n/2+1,...,n])$).
    \end{itemize}
    \item[(c)] Complexity analysis (e.g., via Tree Method, induction, or reusing a bound from class).
\end{itemize}
There exist iterative algorithms for this problem, but the purpose of this exercise is to practice divide \& conquer. Your solution must be a divide \& conquer algorithm, i.e., recursively solving the problem by splitting array $A[1...n]$ into $A[1...n/2]$ and $A[n/2+1...n]$.

For full points, your solution should be as fast as possible.



\section{Median of Medians}
    The standard median of the median algorithm uses groups of five. 
    Instead, consider a median of the median algorithm, with groups of $M \geq 3$. 


    
\begin{enumerate}%[label=(\alph*.)]
    \item 
Give the recurrence for this algorithm (with respect to  $M$) to calculate the worst-case runtime.



\item  Suppose $M=7$, now solve the recurrence by induction. Make sure to clearly write the induction hypothesis, base case, and induction step.

 
 \item  Which of the following is true about the running time of the algorithm in part (b.) for $M=7$ (as compared to $M=5$ from class):
(i) running time improves to $o(n)$, (ii) running time stays $\Theta(n)$, or (iii) running time worsens to $\omega(n)$? Briefly explain your choice.


\end{enumerate}



\end{document}
