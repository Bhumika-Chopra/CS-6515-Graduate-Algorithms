\documentclass[10pt,letterpaper]{article}
\usepackage[utf8]{inputenc}
\usepackage{lmodern} % removes bitmap font
\usepackage{tikz}
\usepackage{nicematrix}
\usepackage[bookmarks,colorlinks,breaklinks]{hyperref}
\hypersetup{urlcolor=blue, colorlinks=true, citecolor=green!50!black, linkcolor=blue}
\usepackage{multirow}
\usepackage[T1]{fontenc} % For polish hook https://tex.stackexchange.com/questions/75396/how-to-use-polishhook-symbol
\usepackage{amsmath}
\usepackage{amsfonts}
\usepackage{amssymb}
\usepackage{amsthm}
\usepackage{thmtools}
\usepackage{pgffor}
\usepackage{xcolor}
\usepackage[lined,boxed,ruled,norelsize,algo2e,linesnumbered,noend]{algorithm2e}


\usepackage{cleveref}
\usepackage{graphicx}
\usepackage{geometry}
\geometry{verbose,tmargin=1in,bmargin=1in,lmargin=1in,rmargin=1in}
\usepackage{enumitem}
\usepackage{tcolorbox}

\declaretheorem[refname={Theorem,Theorems},Refname={Theorem,Theorems}]{theorem}
\declaretheorem[numberlike=theorem]{lemma}
\declaretheorem[numberlike=theorem]{invariant}
\declaretheorem[numberlike=theorem]{corollary}
\declaretheorem[numberlike=theorem]{definition}
\declaretheorem[numberlike=theorem]{claim}
\declaretheorem[numberlike=theorem]{fact}
\declaretheorem[numbered=no]{remark}

% number sets
\newcommand{\R}{\mathbb{R}}
\newcommand{\N}{\mathbb{N}}
\newcommand{\F}{\mathbb{F}}
\newcommand{\Z}{\mathbb{Z}}
\renewcommand{\S}{\mathbb{S}}%

% probability
\renewcommand{\P}{\mathbb{P}}
\newcommand{\E}{\mathbb{E}}

% redfine some classical notation
\renewcommand{\tilde}{\widetilde}
\renewcommand{\hat}{\widehat}
\renewcommand{\bar}{\overline}

\renewcommand{\O}{\mathcal{O}}

% operators
\newcommand{\median}{\operatorname{median}}

\DeclareMathOperator*{\argmin}{argmin}
\DeclareMathOperator*{\argmax}{argmax}
\newcommand{\poly}{\operatorname{poly}}
\newcommand{\polylog}{\operatorname{polylog}}
\newcommand{\mdiag}{\mathbf{Diag}} % diagonal matrix with vector on diag
\newcommand{\sign}{\operatorname{sign}}%
\newcommand{\nnz}{\operatorname{nnz}}%
\newcommand{\tail}{\operatorname{tail}}%
\newcommand{\dist}{\operatorname{dist}}%
\newcommand{\IGNORE}[1]{}

% matrices
% \mX is \mathbf{X}
\foreach \x in {A,...,Z}{%
	\expandafter\xdef\csname m\x\endcsname{\noexpand\mathbf{\x}}
}
% \omX is \overline{\mathbf{X}}
\foreach \x in {A,...,Z}{%
	\expandafter\xdef\csname om\x\endcsname{\noexpand\overline{\noexpand\mathbf{\x}}}
}

\foreach \x in {A,...,Z}{%
	\expandafter\xdef\csname c\x\endcsname{\noexpand\mathcal{\x}}
}

\newcommand{\ob}{\overline{b}}
\newcommand{\ox}{\overline{x}}
\newcommand{\os}{\overline{s}}

% Reduce paragraph spacing https://tex.stackexchange.com/questions/4891/how-do-i-control-the-spacing-above-a-new-paragraph
\makeatletter
\renewcommand{\paragraph}{%
	\@startsection{paragraph}{4}%
	{\z@}{1.25ex \@plus 1ex \@minus .2ex}{-1em}%
	{\normalfont\normalsize\bfseries}%
}
\makeatother

\def\norm#1{\left\| #1 \right\|}



\newcommand{\jan}[1]{\textcolor{red}{Jan: #1}}
\newcommand{\sahil}[1]{\textcolor{blue}{Sahil: #1}}
\newcommand{\dy}[1]{\textcolor{purple}{Daniel Y: #1}}
\newcommand{\mx}[1]{\textcolor{blue}{Max: #1}}
\newcommand{\sg}[1]{\textcolor{cyan}{Sharay: #1}}
\newcommand{\ac}[1]{\textcolor{cyan}{Allen: #1}}
\newcommand{\sera}[1]{\textcolor{magenta}{Sera: #1}}
\newcommand{\frank}[1]{\textcolor{purple}{Frank: #1}}
% \newcommand{\joe}[1]{\textcolor{brown}{Joe: #1}}
\newcommand{\Varun}[1]{\textcolor{brown}{Varun: #1}}
\newcommand{\aryan}[1]{\textcolor{red} {Aryan: #1}}

\title{Linear Programming\\Problem Set 3 -- CS 6515/4540 (Fall 2025)}
\date{}
\begin{document}
\maketitle
\vspace{-50pt}
\noindent
This problem set is due on \textbf{Tuesday October 2nd}. Submission is via Gradescope. Your solution must be a typed pdf (e.g.~via LaTeX) -- no handwritten solutions.



%\paragraph{Additional Instruction:}

%\jan{collecting exercise ideas.}
\setcounter{section}{8}


\section{Convex Sets}
\noindent
We will study properties of convex sets in the Euclidean space. 
\begin{enumerate}
\item Does every convex body $K \subseteq \mathbb{R}^n$ have a finite number of vertices\footnote{Points $u \in K$ is called a vertex if cannot be expressed as a convex combination of some finite number of other points in $K$}? Give a proof or a counterexample. 
\item Is the intersection of two convex bodies also a convex body? What about union? For each, prove it or give a counterexample. 
\item Is a convex body in $n$ dimensions always bounded (finite volume) or could be unbounded? Give a proof that it is bounded or a counterexample. 

 \end{enumerate}


\section{LPs}
    Alice is trying to get enough oranges and bananas to host a fruit party. To successfully host a party she needs \textbf{at least 8 oranges} and \textbf{at least 20 bananas}. Unfortunately, her local grocery story only sells fruit in bundles. Bundle A costs 3 dollars and contains one orange and two bananas. Bundle B costs 2 dollars and contains three oranges and a banana. Fortunately, the grocery story will allow Alice to buy fractions of bundles (i.e. she can buy 2.5 bundle As to get 2.5 oranges and 5 bananas). They will not allow Alice to buy negative bundles (i.e. she cannot buy -1 bundle As and 3 bundle Bs to get 5 oranges and a banana). 

\textbf{Alice would like to buy $x_A$ bundle As and $x_B$ bundle Bs to guarantee she has at least 8 oranges and at least 20 bananas.} Moreover, she would like to  minimize her dollars spent. 

\begin{enumerate}

\item  Write a linear program whose solution is the optimal choice of $x_A, x_B$ for Alice's problem and briefly justify why this is the correct LP.






\item  Show that there exists a solution to this linear program with objective value $30$ (and hence Alice has to spend at most $30$ dollars).




\item  Prove that every feasible solution to this linear program has value at least $30$  (prove this by taking a non-negative linear combination of the constraints), so the solution in the previous part is optimal.

\
\end{enumerate}


\section{LP Equivalent Formulation}
\begin{enumerate}
    \item Show that the class of left LPs (LP1) can efficiently represent any LP on the right (LP2). In other words, if we have a polynomial time algorithm to solve any LP1 on the left, then we can solve in polynomial time  any LP2 on the right.

\begin{minipage}{.5\linewidth}
%\vspace{-0.5cm}
\begin{align*}
\tag{LP1} \max& \sum_{i=1}^n c_i x_i\\
\text{s.t.}\quad \sum_{i=1}^n A_{ji} x_i = b_j, &\quad \forall j \in \{1,\ldots, m\}\\
x_i \geq 0, &\quad \forall i \in \{1,\ldots,n\}.\\
\end{align*}
      \end{minipage}%
 \quad \vrule{}
\begin{minipage}{.5\linewidth}
\begin{align*}
\tag{LP2} \max &\sum_{i=1}^n c_i x_i\\
\text{s.t.}\quad \sum_{i=1}^n A_{ji} x_i \leq b_j, &\quad \forall j \in \{1,\ldots,m\} \\
\end{align*}
      \end{minipage}
(Hint: You need two ideas: (1)~We can replace unconstrained $x$ by $x= x^+ - x^-$ where $x^+ \geq 0$ and  $x^- \geq 0$. (2)~We can replace $Ax \leq b$ by $Ax + Z=b$ for some vector $Z\geq 0$.
)



      
    \item Now use a similar idea to show that Farkas' Lemma A below implies Farkas' Lemma B.
    \begin{itemize}

         \item  Farkas' Lemma A says that the system of inequalities $\sum_{i=1}^n A_{ji} x_i = b_j$ for $j \in \{1,\ldots,m\}$ and $x_i \geq 0$ for $i\in \{1,\ldots, n\}$ are infeasible iff there exist $\lambda_j$ for $j \in \{1,\ldots,m\}$ such that 
        \[
        \sum_{j=1}^m \lambda_j b_j <0 \qquad \text{and} \qquad  
        \sum_{j=1}^m \lambda_j A_{ji} \geq 0  \quad \forall i\in \{1,\ldots, n\}.
        \]

    
        \item  Farkas' Lemma B says that the system of inequalities $\sum_{i=1}^n A_{ji} x_i \leq b_j$ for $j \in \{1,\ldots,m\}$   are infeasible iff there exist \emph{non-negative} $\lambda_j \geq 0$ for $j \in \{1,\ldots,m\}$ such that 
        \[
        \sum_{j=1}^m \lambda_j b_j < 0 \qquad \text{and} \qquad  
        \sum_{j=1}^m \lambda_j A_{ji} = 0  \quad \forall i\in \{1,\ldots, n\}.
        \]
        \end{itemize}

    
\end{enumerate}


\section{LP for Regression}
%\begin{enumerate}     \item 
Suppose we are given $n$ points $(x_1, y_1),(x_2, y_2),...,(x_n, y_n) \in \mathbb{R}\times \mathbb{R}$. The goal of this problem is to find a line $y = ax + b$, where $a,b \in\mathbb{R}$   that fits these points as closely as possible, where closeness is defined according to some objective function.
Write a polynomial-sized LP for the following settings:
\begin{enumerate}
    \item $\ell_1$ regression: Objective is to minimize $ \sum_{i=1}^n |y_i - a x_i -b|$.
    
  
    
    \item $\ell_\infty$ regression: Objective is to  minimize  $ \max_{i=1}^n |y_i - a x_i -b|$

 

    \item Write the dual LPs for both the above LPs.
\end{enumerate}


\end{document}