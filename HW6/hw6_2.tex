\documentclass[10pt,letterpaper]{article}
\usepackage[utf8]{inputenc}
\usepackage{lmodern} % removes bitmap font
\usepackage{tikz}
\usepackage{nicematrix}
\usepackage[bookmarks,colorlinks,breaklinks]{hyperref}
\hypersetup{urlcolor=blue, colorlinks=true, citecolor=green!50!black, linkcolor=blue}
\usepackage{multirow}
\usepackage[T1]{fontenc} % For polish hook https://tex.stackexchange.com/questions/75396/how-to-use-polishhook-symbol
\usepackage{amsmath}
\usepackage{amsfonts}
\usepackage{amssymb}
\usepackage{amsthm}
\usepackage{thmtools}
\usepackage{pgffor}
\usepackage{xcolor}
\usepackage[lined,boxed,ruled,norelsize,algo2e,linesnumbered,noend]{algorithm2e}


\usepackage{cleveref}
\usepackage{graphicx}
\usepackage{geometry}
\geometry{verbose,tmargin=1in,bmargin=1in,lmargin=1in,rmargin=1in}
\usepackage{enumitem}
\usepackage{tcolorbox}

\declaretheorem[refname={Theorem,Theorems},Refname={Theorem,Theorems}]{theorem}
\declaretheorem[numberlike=theorem]{lemma}
\declaretheorem[numberlike=theorem]{invariant}
\declaretheorem[numberlike=theorem]{corollary}
\declaretheorem[numberlike=theorem]{definition}
\declaretheorem[numberlike=theorem]{claim}
\declaretheorem[numberlike=theorem]{fact}
\declaretheorem[numbered=no]{remark}

% number sets
\newcommand{\R}{\mathbb{R}}
\newcommand{\N}{\mathbb{N}}
\newcommand{\F}{\mathbb{F}}
\newcommand{\Z}{\mathbb{Z}}
\renewcommand{\S}{\mathbb{S}}%

% probability
\renewcommand{\P}{\mathbb{P}}
\newcommand{\E}{\mathbb{E}}

% redfine some classical notation
\renewcommand{\tilde}{\widetilde}
\renewcommand{\hat}{\widehat}
\renewcommand{\bar}{\overline}

\renewcommand{\O}{\mathcal{O}}

% operators
\newcommand{\median}{\operatorname{median}}

\DeclareMathOperator*{\argmin}{argmin}
\DeclareMathOperator*{\argmax}{argmax}
\newcommand{\poly}{\operatorname{poly}}
\newcommand{\polylog}{\operatorname{polylog}}
\newcommand{\mdiag}{\mathbf{Diag}} % diagonal matrix with vector on diag
\newcommand{\sign}{\operatorname{sign}}%
\newcommand{\nnz}{\operatorname{nnz}}%
\newcommand{\tail}{\operatorname{tail}}%
\newcommand{\dist}{\operatorname{dist}}%
\newcommand{\IGNORE}[1]{}

% matrices
% \mX is \mathbf{X}
\foreach \x in {A,...,Z}{%
	\expandafter\xdef\csname m\x\endcsname{\noexpand\mathbf{\x}}
}
% \omX is \overline{\mathbf{X}}
\foreach \x in {A,...,Z}{%
	\expandafter\xdef\csname om\x\endcsname{\noexpand\overline{\noexpand\mathbf{\x}}}
}

\foreach \x in {A,...,Z}{%
	\expandafter\xdef\csname c\x\endcsname{\noexpand\mathcal{\x}}
}

\newcommand{\ob}{\overline{b}}
\newcommand{\ox}{\overline{x}}
\newcommand{\os}{\overline{s}}

% Reduce paragraph spacing https://tex.stackexchange.com/questions/4891/how-do-i-control-the-spacing-above-a-new-paragraph
\makeatletter
\renewcommand{\paragraph}{%
	\@startsection{paragraph}{4}%
	{\z@}{1.25ex \@plus 1ex \@minus .2ex}{-1em}%
	{\normalfont\normalsize\bfseries}%
}
\makeatother

\def\norm#1{\left\| #1 \right\|}



\newcommand{\jan}[1]{\textcolor{red}{Jan: #1}}
\newcommand{\sahil}[1]{\textcolor{blue}{Sahil: #1}}
\newcommand{\dy}[1]{\textcolor{purple}{Daniel Y: #1}}
\newcommand{\mx}[1]{\textcolor{blue}{Max: #1}}
\newcommand{\sg}[1]{\textcolor{cyan}{Sharay: #1}}
\newcommand{\ac}[1]{\textcolor{cyan}{Allen: #1}}
\newcommand{\sera}[1]{\textcolor{magenta}{Sera: #1}}
\newcommand{\frank}[1]{\textcolor{purple}{Frank: #1}}
% \newcommand{\joe}[1]{\textcolor{brown}{Joe: #1}}
\newcommand{\Varun}[1]{\textcolor{brown}{Varun: #1}}
\newcommand{\aryan}[1]{\textcolor{red} {Aryan: #1}}

\title{Online Convex Optimization \& Complexity\\Problem Set 6 -- CS 6515/4540 (Fall 2025)}
\date{}
\begin{document}
\maketitle
\vspace{-50pt}
\noindent
This problem set is due on \textbf{Monday November 17th}. Submission is via Gradescope. Your solution must be a typed pdf (e.g.~via LaTeX) -- no handwritten solutions.



%\paragraph{Additional Instruction:}

%\jan{collecting exercise ideas.}
\setcounter{section}{21}

\section{Online Gradient Descent}
Online gradient descent is frequently utilized with stochastic gradient descent. Suppose that we want to run online gradient descent but cannot query the gradient of our convex function $f_t$, however we can construct a random vector variable $G_t$ that estimates $\nabla f_t$ at any $x_t$: \[\nabla f_t(x_t) = \mathbb{E}[G_t]\]
Show that if we run Online Gradient Descent using $G_t$ as our gradient of $f_t$ at time $t$, then we have \[\mathbb{E}\left[\sum_{t = 0}^{T}f_t(x_t) - f_t(x^*)\right] \leq DG\sqrt{T}\] where $G$ in this case will satisfy $\mathbb{E}[||G_t||^2] \leq G^2$ for all $t$. 


\section{Missing Proof Details}
\begin{enumerate}
    \item In Lecture 20 we assumed that the following convex body $K$ has diameter $O(\sqrt{n})$. Prove that.
    \[K = \{ x\in \mathbb{R}^{n\times n}_{\geq 0} \mid \sum_{i} x_{ij}=1 \quad \forall j \quad \text{and}  \quad \sum_{j} x_{ij}=1 \quad  \forall i\}. \]
    

    \item In Lecture 21 while proving the minimax theorem, by applying OGD and taking $T$ large enough, we obtained
    \[
    \frac{1}{T} \sum_t M(x_t,a_t) \geq \frac{1}{T}\max_i \sum_t M(i,a_t) - \epsilon.
    \]
    Prove why the LHS is at most expression B and why the RHS is at least expression A$-\epsilon$ to complete the proof of the minimax theorem.

    \item Suppose you run online gradient descent in $n$-dimensions where for all $t$,  $f_t(x)=\sum_{i=1}^n a^t_i x_i$ for some non-negative reals $a^t_i \in [0,A]$ and where convex body $K=\{ x\in \mathbb{R}^n_{\geq 0} \mid \sum_{i} x_i = B \}$ for some positive $B$. After $T$ iterations of online gradient descent, what is the best possible upper bound on the regret of the algorithm? (Give the answer in Big-O notation in terms of parameters $A,B$, and $n$.) 
\end{enumerate}




\section{Zero Sum Games}



  Consider a  zero-sum game between players A and B with   A having actions $\{A_1,A_2\}$ and B having actions $\{B_1,B_2\}$.  The  entries in this table correspond to the rewards of  player A (i.e., the costs of player B):

\begin{center}
\begin{tabular}{c|c|c} 
  & $B_1$ & $B_2$ \\ 
  \hline
 $A_1$ & 2 & -1 \\ 
 \hline
 $A_2$& -1& 1  \\ 
\end{tabular}
\end{center}

\begin{enumerate}
    \item If  player A has to commit a strategy before  player B (i.e.,  B will choose their best strategy/action knowing A's randomized strategy),  what strategy would they adopt and how much expected reward would it get? 

(Hint: First solve for a general randomized strategy playing $A_1$ with probability $p$ and $A_2$ with probability $1-p$, and then optimize $p$.)

\item Now answer the above question when player B has to first commit a strategy (and then player A chooses their best strategy/action): what strategy would B adopt and how much would be the expected cost? 

\item Are the answers to both the above parts the same? Why does it agree with the minimax theorem?
\end{enumerate}



\section{Reductions: Dominating Set}
Given a graph $G=(V,E)$, dominating set $S \subseteq V$ is a subset of vertices such that every vertex $v\in V$ has at least one neighbor in $S$. Dominating set problem asks to find the minimum dominating set of $G$. 

\begin{enumerate}
    \item Prove that an $\alpha$-approx algorithm for set cover implies an  $\alpha$-approx algorithm for dominating set.
    

    \item Prove that an $\alpha$-approx algorithm for dominating implies an $\alpha$-approx algorithm for set cover.
    
\end{enumerate}


\section{Fine-Grained Complexity}
Consider two variants of orthogonal vectors problem.
\begin{itemize}
    \item Problem A: Given set $S$ of $n$ vectors $s_1,\ldots, s_n$ in $\{0,1\}^d$, find two vectors $s_i,s_j$ that are orthogonal to each other.

\item Problem B: Given two sets $V$, $W$  of $n/2$ vectors each in $\{0,1\}^d$, where $V=\{v_1,\ldots, v_{n/2}\}$ and $W=\{w_1,\ldots, w_{n/2} \}$, find two vectors $v_i,w_j$ that are orthogonal to each other.
\end{itemize}

Prove that these two problems are equivalent in the sense that  there exists $O(n^{2-\delta}d)$ runtime algorithm for one, where $\delta>0$ is some constant, iff there exists $O(n^{2-\delta}d)$ runtime algorithm for the other one.
\appendix

\section{Further Practice Problems}


\subsection{Experts Problem}
Prove that the greedy algorithm, which in each round $t\in [T]$ plays the best expert based on performance in rounds $\{1,\ldots, t-1\}$, could incur $\Omega(T)$ total regret.

(Hint: There is an example already with $n=2$ experts.)












\end{document}