\documentclass[10pt,letterpaper]{article}
\usepackage[utf8]{inputenc}
\usepackage{lmodern} % removes bitmap font
\usepackage{tikz}
\usepackage{nicematrix}
\usepackage[bookmarks,colorlinks,breaklinks]{hyperref}
\hypersetup{urlcolor=blue, colorlinks=true, citecolor=green!50!black, linkcolor=blue}
\usepackage{multirow}
\usepackage[T1]{fontenc} % For polish hook https://tex.stackexchange.com/questions/75396/how-to-use-polishhook-symbol
\usepackage{amsmath}
\usepackage{amsfonts}
\usepackage{amssymb}
\usepackage{amsthm}
\usepackage{thmtools}
\usepackage{pgffor}
\usepackage{xcolor}
\usepackage[lined,boxed,ruled,norelsize,algo2e,linesnumbered,noend]{algorithm2e}


\usepackage{cleveref}
\usepackage{graphicx}
\usepackage{geometry}
\geometry{verbose,tmargin=1in,bmargin=1in,lmargin=1in,rmargin=1in}
\usepackage{enumitem}
\usepackage{tcolorbox}

\declaretheorem[refname={Theorem,Theorems},Refname={Theorem,Theorems}]{theorem}
\declaretheorem[numberlike=theorem]{lemma}
\declaretheorem[numberlike=theorem]{invariant}
\declaretheorem[numberlike=theorem]{corollary}
\declaretheorem[numberlike=theorem]{definition}
\declaretheorem[numberlike=theorem]{claim}
\declaretheorem[numberlike=theorem]{fact}
\declaretheorem[numbered=no]{remark}

% number sets
\newcommand{\R}{\mathbb{R}}
\newcommand{\N}{\mathbb{N}}
\newcommand{\F}{\mathbb{F}}
\newcommand{\Z}{\mathbb{Z}}
\renewcommand{\S}{\mathbb{S}}%

% probability
\renewcommand{\P}{\mathbb{P}}
\newcommand{\E}{\mathbb{E}}

% redfine some classical notation
\renewcommand{\tilde}{\widetilde}
\renewcommand{\hat}{\widehat}
\renewcommand{\bar}{\overline}

\renewcommand{\O}{\mathcal{O}}

% operators
\newcommand{\median}{\operatorname{median}}

\DeclareMathOperator*{\argmin}{argmin}
\DeclareMathOperator*{\argmax}{argmax}
\newcommand{\poly}{\operatorname{poly}}
\newcommand{\polylog}{\operatorname{polylog}}
\newcommand{\mdiag}{\mathbf{Diag}} % diagonal matrix with vector on diag
\newcommand{\sign}{\operatorname{sign}}%
\newcommand{\nnz}{\operatorname{nnz}}%
\newcommand{\tail}{\operatorname{tail}}%
\newcommand{\dist}{\operatorname{dist}}%
\newcommand{\IGNORE}[1]{}

% matrices
% \mX is \mathbf{X}
\foreach \x in {A,...,Z}{%
	\expandafter\xdef\csname m\x\endcsname{\noexpand\mathbf{\x}}
}
% \omX is \overline{\mathbf{X}}
\foreach \x in {A,...,Z}{%
	\expandafter\xdef\csname om\x\endcsname{\noexpand\overline{\noexpand\mathbf{\x}}}
}

\foreach \x in {A,...,Z}{%
	\expandafter\xdef\csname c\x\endcsname{\noexpand\mathcal{\x}}
}

\newcommand{\ob}{\overline{b}}
\newcommand{\ox}{\overline{x}}
\newcommand{\os}{\overline{s}}

% Reduce paragraph spacing https://tex.stackexchange.com/questions/4891/how-do-i-control-the-spacing-above-a-new-paragraph
\makeatletter
\renewcommand{\paragraph}{%
	\@startsection{paragraph}{4}%
	{\z@}{1.25ex \@plus 1ex \@minus .2ex}{-1em}%
	{\normalfont\normalsize\bfseries}%
}
\makeatother

\def\norm#1{\left\| #1 \right\|}



\newcommand{\jan}[1]{\textcolor{red}{Jan: #1}}
\newcommand{\sahil}[1]{\textcolor{blue}{Sahil: #1}}
\newcommand{\dy}[1]{\textcolor{purple}{Daniel Y: #1}}
\newcommand{\mx}[1]{\textcolor{blue}{Max: #1}}
\newcommand{\sg}[1]{\textcolor{cyan}{Sharay: #1}}
\newcommand{\ac}[1]{\textcolor{cyan}{Allen: #1}}
\newcommand{\sera}[1]{\textcolor{magenta}{Sera: #1}}
\newcommand{\frank}[1]{\textcolor{purple}{Frank: #1}}
% \newcommand{\joe}[1]{\textcolor{brown}{Joe: #1}}
\newcommand{\Varun}[1]{\textcolor{brown}{Varun: #1}}
\newcommand{\aryan}[1]{\textcolor{red} {Aryan: #1}}

\title{Convex Programming\\Problem Set 5 -- CS 6515/4540 (Fall 2025)}
\date{}
\begin{document}
\maketitle
\vspace{-50pt}
\noindent
This problem set is due on \textbf{Tuesday November 4th}. Submission is via Gradescope. Your solution must be a typed pdf (e.g.~via LaTeX) -- no handwritten solutions.



%\paragraph{Additional Instruction:}

%\jan{collecting exercise ideas.}
\setcounter{section}{17}

\section{Convex Functions}
\begin{enumerate} 

\item Given two  convex functions $f : \mathbb{R} \rightarrow \mathbb{R} $ and $g : \mathbb{R} \rightarrow \mathbb{R}$, is the sum $f + g$ also convex? Either prove it or give a counterexample. 


\item Given two  convex functions $f : \mathbb{R} \rightarrow \mathbb{R} $ and $g : \mathbb{R} \rightarrow \mathbb{R}$, is the product $fg$ also convex? Either prove it or give a counterexample. 
\item What about the convexity of the function $h : \mathbb{R} \rightarrow \mathbb{R}$ defined by $h(x) = \max (f(x), g(x))$? Prove it or give a counterexample. 
\item Show an $\alpha$-strongly convex function (defined in Pb \ref{sec:StrongConvex}) $f : \mathbb{R} \rightarrow \mathbb{R}$ satisfies $f(x) = \Omega(x^2)$
\end{enumerate}

\section{Gradient Descent Failure}
 Suppose we run an unconstrained Gradient Descent on $f(x) = \frac{1}{2} x^2$ with some arbitrary step size.  Give (and justify) an example consisting of \begin{enumerate} \item  a step size $\eta > 0$ \item an initial point $x_0 \in \mathbb{R}$ \end{enumerate}such that  $t$-th \textbf{average} $\overline{x}_t = \frac{1}{t}\sum_{i \leq t}x_i$ of an unconstrained Gradient Descent with the above parameters does not converge (e.g it diverges) to the optimum  as $t \rightarrow \infty$. 

\section{Gradient Descent for Strongly-Convex Functions} \label{sec:StrongConvex}
 A differentiable function $f$ is $\alpha$-strongly convex for $\alpha >0$ if for all $x,y \in \mathbb{R}^n$ we have 
 \[ f(y) \geq f(x) + \langle \nabla f(x), y-x \rangle + \frac{\alpha}{2}\|y-x\|_2^2 .\]
 
 Consider an $\alpha$-strongly convex differentiable function $f$  with the 2-norm of its gradient always bounded by $G$. The goal is to minimize $f$ and let $x^*$ denote its minimum. 
 
 Show that the  gradient descent algorithm with  step size $\frac{1}{\alpha (t+1)}$ satisfies 
 \[ f\Big( \frac{\sum_t x_t}{T} \Big) - f(x^*) \leq \frac{G^2 (1+\log T)}{2 \alpha T}.\] 
 Thus, strong-convexity allows us to get $1/T$ dependency in regret instead of $1/\sqrt{T}$ dependency for general convex functions.

(Hint: Change the potential function in the analysis from class to  $\Phi(t) = \frac{t \alpha}{2} \|x_t - x^* \|^2$. Also, use that $\sum_{t\in \{1,\ldots, T\}}\frac{1}{t} \leq 1+\log T$.)



\section{Non-Convex Function}

In class we assumed function $f$ is convex. We now want to consider the non-convex case.
We want to show that for $L$-smooth $f$, after $t$ iterations with step size $\eta \le 1/L$ we can find a point $x'$ with
$$
\|\nabla f(x')\| \le \sqrt{\frac{2}{\eta \cdot t} (f(x^0) - f(x^*))}.
$$
(Note that for a local optimum we have $\nabla f(x) = 0$, so a small norm $\|\nabla f(x')\|$ indicates that we are close to a local optimum or saddle point.)

%Assume $f$ is an $L$-smooth function. (Not necessarily convex.)
Proving this from scratch is a bit tricky, so we provide the following subproblems to guide you to a proof. Each subproblem can be solved in a few lines of calculation/algebra.
\paragraph{Problem:}
\begin{enumerate}
    %\item Show
    %$f(x+\delta) \le f(x) + \nabla f(x)^\top \delta + \frac{L}{2}\|\delta\|^2$
    \item Show
    $f(x^{t+1}) \le f(x^t) - \frac{\eta}{2} \|\nabla f(x^t)\|^2$ (Hint: Check the proof from class for convex functions. Does it work for non-convex functions?)
    \item Show
    $
    \sum_{k=0}^t \|\nabla f(x^k)\|^2 \le \frac{2}{\eta} (f(x^0) - f(x^*))
    $ (Hint: 1.~implies $\frac{\eta}{2} \|\nabla f(x^t)\|^2 \le ...$)
    \item Show
    $
\min_{k=0...t}\|\nabla f(x^k)\| \le \sqrt{\frac{2}{\eta \cdot t} (f(x^0) - f(x^*))}.
$
\end{enumerate}
where $x^*$ is the global optimum $f(x^*) = \min_x f(x)$.

You are allowed to use subproblems to solve later subproblems (e.g., use 1+2 to solve 3), even if you did not prove them. 



\end{document}